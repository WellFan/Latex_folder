\documentclass[12pt]{article}
\usepackage{diagbox}
\usepackage{amsmath}
\usepackage{graphicx} %插入图片的宏包
\usepackage{float} %设置图片浮动位置的宏包
\usepackage{subfigure} %插入多图时用子图显示的宏包
\usepackage{setspace}
\usepackage{xeCJK}
\usepackage{amssymb}
\usepackage{biblatex}
\usepackage{enumerate}
\usepackage{multirow}
\usepackage[rightcaption]{sidecap}
\usepackage{caption}
\usepackage{fontspec}
%%%%%%%%%%%%%%%%%%%%%%%%%%%%%%%%%%%%%%%%%%%%%%%%%%%%%%%%%%%%%%%%%%%%%%%%%%%%
\setCJKmainfont{NotoSerifTC-Regular.otf} %自行去 google font 下載該字型
\XeTeXlinebreaklocale "zh"             %這兩行一定要加,中文才能自動換行
\XeTeXlinebreakskip = 0pt plus 1pt     %這兩行一定要加,中文才能自動換行
\defaultCJKfontfeatures{AutoFakeBold=6,AutoFakeSlant=.4} %以後不用再設定粗斜
\newCJKfontfamily\Kai{標楷體}           %定義指令\Kai則切換成標楷體
\newCJKfontfamily\Hei{微軟正黑體}       %定義指令\Hei則切換成正黑體
\newCJKfontfamily\NewMing{新細明體}     %定義指令\NewMing則切換成新細明體
\doublespacing
%%%%%%%%%%%%%%%%%%%%%%%%%%%%%%%%%%%%%%%%%%%%%%%%%%%%%%%%%%%%%%%%%%%%%%%%%%%%
\title{Application of Nested Logit Model on Recommendation System}
\author{Wei-Yu Fan\thanks{
Graduate student in Department of Economics, National Taiwan University.\\ 
Email address: entrencemania@gmail.com
}, Yu-Chan Chen
} 
\date{May 2024}
%%%%%%%%%%%%%%%%%%%%%%%%%%%%%%%%%%%%%%%%%%%%%%%%%%%%%%%%%%%%%%%%%%%%%%%%%%%%
\begin{document}
\maketitle
\begin{spacing}{0}
\begin{abstract}\noindent
我們探討了 nested logit model 在推薦系統上的應用。我們透過隨機效用模型與 nested logit distribution 將消費者與商品的關係建模,在有限的廣告欄位下決定商品的組合並推播給消費者以最大化消費者的點籍機率。並且我們推導出此模型有一個 closed-form solution,並提供與之對應的演算法。
\end{abstract}
\end{spacing}
\begin{tabular}{rl}
\\
\textbf{Keywords:} &Random Utility Model, Nested Logit Model, \\
&Recommendation System
\end{tabular}

\newpage


\section{Introduction}

\section{Related Literature}

\section{Methodologies}

\section{Analysis}

\section{Conclusion}

\begin{thebibliography}{1}
\bibitem{Irandoust, Manuchehr}
    Irandoust, Manuchehr. (2019).\emph{House Prices and Unemployment: An Empirical Analysis of Causality},
    International Journal of Housing Markets and Analysis. 12. 148-164. 

\bibitem{chu}
    Shiou-Yen Chu (2018).\emph{Macroeconomic policies and housing market in Taiwan},
    International Review of Economics \& Finance,Volume 58, 404-421.
\end{thebibliography}



\end{document}