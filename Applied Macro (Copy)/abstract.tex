\documentclass[12pt]{article}
\usepackage[ top=1.5cm]{geometry}
\usepackage{graphicx} % Required for inserting images
\usepackage{amsmath}
\usepackage{graphicx} %插入图片的宏包
\usepackage{float} %设置图片浮动位置的宏包
\usepackage{subfigure} %插入多图时用子图显示的宏包
\usepackage{setspace}
\usepackage{xeCJK}
\usepackage{amsmath}
\usepackage{amssymb}
\usepackage{biblatex}
\usepackage{enumerate}
\usepackage{multirow}
\usepackage{graphicx}
\usepackage{float}
\usepackage{subfigure}
\usepackage[rightcaption]{sidecap}

\title{ Time-Varying Relationship between Taiwan's Unemployment Rate and Housing Prices }
\author{Ting-Chun Lin\thanks{
Undergraduate student in Department of Economics, National Taiwan University.\\ Email address: b09303097@ntu.edu.tw
}
} 
\linespread{2}
\date{April 2023}


\begin{document}

\maketitle

\begin{abstract}
    This paper examines the relationship between high housing price and low unemployment rate in Taiwan. In this studies, we built a ARDL model to capture the effect of unemployment rate on housing price while holding Gross domestic product (GDP) and interest rate as control variables. Through inspecting data during 2000Q1-2022Q4, our studies discover that during different time period, unemployment rate has different effects on the housing price. The empirical results show that when the macro condition unpredictably becomes dreadful, the effect of unemployment rate significantly influenced housing price.
\end{abstract}
\end{document}