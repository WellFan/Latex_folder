\documentclass{beamer}
\usepackage[backend=biber, style=authoryear,]{biblatex}
\usepackage{tikz}
\usepackage{amsmath}
\usepackage{amssymb}
\usepackage{amsthm}
\usepackage{multirow}
\usepackage{caption}
\usepackage{diagbox}
\usetheme{Madrid}
\usecolortheme{}
\newcommand{\matr}[1]{\mathbf{#1}}
%Information to be included in the title page:
\title[Oil, FX swaps and interst rate]{Further Exploration of "Oil, Foreign Exchange Swaps and Interest Rates in the GCC Countries"}
\author{Wei-Yu, Fan}
\institute[NTU]{Department of Economics, NTU}
\date{June 2024}
\addbibresource{reference.bib}

\begin{document}

\frame{\titlepage}

\begin{frame}{Introduction}
In \textcite{almaskati2022oil} study, it was mentioned that there is a correlation between oil prices, foreign exchange (FX) swaps, and interest rates among the member countries of the Gulf Cooperation Council (GCC).
\end{frame}

\begin{frame}{Introduction}
The possible reasons for the correlation are:
\begin{itemize}
    \item The majority of their foreign exchange comes from revenue generated by oil exports
    \item Oil exports constitute a significant portion of GDP and government revenue, and government entities, public investment funds, and sovereign wealth funds
    \item Those entities are major liquidity providers for local banks
\end{itemize}
\end{frame}

\begin{frame}{Related Literature}
\begin{itemize}
    \item \textcite{amano1998oil} observed that the trends in oil prices significantly impact the real exchange rate of the United States.
    \item \textcite{zhang2008spillover} noted a long-term correlation between oil prices and the US dollar exchange rate, suggesting that it is the US dollar exchange rate that influences oil prices.
    \item \textcite{reboredo2014oil} found a long-term contagion effect between oil prices and the US dollar exchange rate, exacerbating the 2008 global financial crisis.
    \item \textcite{eslamloueyan2015determinants} discovered that oil prices play a crucial role in determining the real exchange rates of GCC member countries in the long term.
\end{itemize}
\end{frame}

\begin{frame}{Data decription}
We extracted the daily closing prices of several financial assets from Datastream during the period from May 2009 to May 2024. The data includes:
\begin{itemize}
    \item The 3-month FX swaps of GCC countries and euro
    \item The 3-month interbank offered rates of United Arab Emirates (EIBO3M) and the London (USDL3M)
    \item The daily closing prices of the 3-month Kuwaiti dinar, Omani rial, Qatari riyal, Saudi riyal, and United Arab Emirates dirham FX swaps
    \item Spot and futures commodity prices (oil/gold/gas) 
    \item SP 500 U.S. stock index (SPX) and MSCI Emerging Markets Index (MSCI-EM)
\end{itemize}
Then we calculated and use the daily returns of these assets.
\end{frame}

\begin{frame}{Methodology}
We regress oil prices, FX swap, local interbank offered rates, SP500, gold prices, and LIBOR using a vector autoregression model (VAR).

The model is as follows:
\begin{equation*}
    \begin{aligned}
    &\matr{y}_t := (\Delta \log \text{FXswap}_t, \Delta \log \text{r}_t, \Delta \log \text{Oil}_t, ...)^T \in \mathbb{R}^p, \\
    &\matr{y}_t = \matr{A}_{-1}\matr{y}_{t-1}+\matr{A}_{-2}\matr{y}_{t-2}+...+\matr{A}_{-k}\matr{y}_{t-k}+e_t.
    \end{aligned}
\end{equation*}

\end{frame}

\begin{frame}{Data Analysis}
    \begin{table}
        \centering
        \caption{Summary Statistics of the Return rates \footnote[1]{Jarque-Bera statistic, Augmented Dickey-Fuller (ADF) test, and Phillips-Perron (PP) test for each variable. The Jarque-Bera statistic tests the null hypothesis that the data is normally distributed, while the ADF and PP tests examine the null hypothesis that the data has a unit root.}}
        \label{table1}
        \resizebox{\columnwidth}{!}{%
        \begin{tabular}{|lrrrrrrrrr|}
            \hline
            Name & Mean & Maximun & Minimun & Std. Dev. & Skewness & Kurtosis & Jarque-Bera & ADF & PP \\
            \hline
            OMR3M & 0.000368 & 4.580394 & -3.126833 & 0.321510 & 1.127371 & 33.667119 & 0.000000 & 0.000000 & 0.000000 \\
            BHD3M & 0.000178 & 1.471405 & -1.590376 & 0.152445 & -0.123410 & 16.192026 & 0.000000 & 0.000000 & 0.000000 \\
            KWD3M & -0.000181 & 2.921430 & -2.821397 & 0.241882 & -0.075060 & 18.478365 & 0.000000 & 0.000000 & 0.000000 \\
            QAR3M & 0.000023 & 5.930881 & -5.955600 & 0.300399 & -1.029339 & 125.294164 & 0.000000 & 0.000000 & 0.000000 \\
            SAR3M & 0.000030 & 0.435031 & -0.461680 & 0.025713 & 0.564578 & 75.888848 & 0.000000 & 0.000000 & 0.000000 \\
            AED3M & -0.000012 & 0.295667 & -0.258926 & 0.012378 & 0.727567 & 171.852696 & 0.000000 & 0.000000 & 0.000000 \\
            EUR3M & -0.005693 & 2.567288 & -2.242401 & 0.527834 & 0.047230 & 1.589542 & 0.000000 & 0.000000 & 0.000000 \\
            EIBO3M & 0.019403 & 44.497508 & -49.061664 & 4.340056 & -0.578778 & 33.842150 & 0.000000 & 0.000000 & 0.000000 \\
            Gold & 0.023873 & 5.133427 & -9.596165 & 0.971210 & -0.455444 & 5.634502 & 0.000000 & 0.000000 & 0.000000 \\
            Oil & -0.020409 & 123.677917 & -305.966065 & 5.886999 & -33.463558 & 1913.033934 & 0.000000 & 0.000000 & 0.000000 \\
            MSCI EM & 0.011127 & 5.581817 & -6.943303 & 1.026245 & -0.399815 & 3.778718 & 0.000000 & 0.000000 & 0.000000 \\
            SPX & 0.045251 & 8.968316 & -12.765214 & 1.081019 & -0.707849 & 13.183011 & 0.000000 & 0.000000 & 0.000000 \\
            USDL3M & 0.047995 & 24.885484 & -27.263636 & 1.702867 & -0.801945 & 48.911574 & 0.000000 & 0.000000 & 0.000000 \\
            GAS & -0.032772 & 38.172675 & -69.314718 & 3.662205 & -1.330571 & 39.500504 & 0.000000 & 0.000000 & 0.000000 \\
            \hline
        \end{tabular}}%}
    \end{table}
\end{frame}

\begin{frame}{Data Analysis}
    \begin{table}
        \centering
        \caption{Correlation of the Return rates}
        \label{table2}
        \resizebox{\columnwidth}{!}{%
        \begin{tabular}{|lrrrrrrrrrrrrrr|}
            \hline
             & OMR3M & BHD3M & KWD3M & QAR3M & SAR3M & AED3M & EUR3M & EIBO3M & Gold & Oil & MSCI EM & SPX & USDL3M & GAS \\
            \hline
            OMR3M & 1.000000 & &  &  &  &  &  &  &  &  &  &  &  &  \\
            BHD3M & 0.064175 & 1.000000 &  &  & &  &  &  &  &  &  &  & &  \\
            KWD3M & -0.004342 & 0.048295 & 1.000000 &  & & &  &  &  &  &  &  &  & \\
            QAR3M & -0.032026 & 0.014369 & 0.021553 & 1.000000 &  & &  &  &  &  &  &  &  &  \\
            SAR3M & 0.075297 & 0.038707 & 0.016946 & 0.024471 & 1.000000 & &  &  &  &  &  &  &  &  \\
            AED3M & 0.073529 & 0.018838 & 0.087568 & 0.002695 & 0.064088 & 1.000000 & &  &  &  &  &  &  &  \\
            EUR3M & 0.007711 & 0.010918 & -0.202587 & 0.010328 & 0.000726 & -0.025512 & 1.000000 &  &  &  & & &  &  \\
            EIBO3M & -0.011979 & -0.014450 & -0.010113 & -0.013793 & 0.013951 & -0.004707 & 0.021393 & 1.000000 & &  & &  &  &  \\
            Gold & 0.037293 & 0.004764 & -0.084953 & 0.002930 & 0.003426 & -0.016769 & 0.363884 & 0.023088 & 1.000000 & & &  & &  \\
            Oil & 0.058072 & 0.036930 & -0.018069 & -0.002890 & 0.082453 & 0.014696 & 0.049036 & -0.000047 & 0.045790 & 1.000000 &  &  & &  \\
            MSCI EM & -0.000011 & -0.000820 & -0.064095 & -0.016388 & -0.003269 & 0.012727 & 0.292853 & -0.004832 & 0.210766 & 0.100368 & 1.000000 &  &  &  \\
            SPX & 0.047305 & -0.009077 & -0.067865 & -0.029824 & 0.027595 & -0.007703 & 0.162321 & -0.009381 & 0.068460 & 0.138281 & 0.453554 & 1.000000 & &  \\
            USDL3M & 0.042462 & -0.031377 & -0.007134 & -0.033063 & -0.000350 & -0.014125 & -0.070792 & -0.018465 & -0.078611 & -0.023823 & -0.041639 & 0.009709 & 1.000000 & \\
            GAS & -0.012172 & 0.004382 & -0.025919 & 0.024487 & 0.017694 & -0.015924 & 0.024572 & -0.034489 & 0.000870 & 0.005898 & 0.042040 & 0.070901 & -0.025984 & 1.000000 \\
            \hline
        \end{tabular}}%}
    \end{table}
\end{frame}

\begin{frame}{Results}
\begin{table}
    \centering
    \caption{VAR Model Results($\alpha$ = 5\%)\footnote[2]{We find that the AIC and BIC results are quite different. The AIC suggests a lag of 9, while the BIC suggests a lag of 2. We choose the lag of 2 as it is more conservative.}}
    \label{table3}
    \resizebox{\columnwidth}{!}{%
    \begin{tabular}{lrrrrrrrrrrrrrr}
        \hline
         & OMR3M & BHD3M & KWD3M & QAR3M & SAR3M & AED3M & EIBO3M  & Oil & USDL3M\\
        \hline
        L1.OMR3M & -0.388544 & 0.026061 & 0.035832 &  & 0.002620 &  &  & -0.773951 &  \\
        L1.BHD3M & 0.093795 & -0.415733 &  &  &  &  &  &  &  \\
        L1.KWD3M & 0.043305 &  & -0.258733 &  &  & -0.001989 &  &  & 0.320108 \\
        L1.QAR3M & 0.036584 &  &  & -0.445284 &  &  & 0.469332 &  &  \\
        L1.SAR3M & 0.584286 & 0.256249 &  &  & -0.428796 & 0.016081 &  &  &  \\
        L1.AED3M &  &  & -1.053435 &  &  & -0.559541 &  &  &  \\
        L1.EIBO3M & -0.002523 &  &  &  &  &  & -0.367572 &  & 0.027283 \\
        L1.Oil &  &  &  &  &  &  &  & -0.253198 &  \\
        L1.USDL3M &  &  &  &  &  &  & 0.204014 &  & 0.132099\\
        L2.OMR3M & -0.092432 &  & 0.039735 &  &  &  & -0.463412 & -1.363542 &  \\
        L2.BHD3M &  & -0.171601 &  &  &  &  &  &  &  \\
        L2.KWD3M & 0.048206 & 0.021473 & -0.093834 &  & 0.004291 &  &  &  &  \\
        L2.QAR3M &  &  &  & -0.159426 &  &  &  & -0.854648 &  \\
        L2.SAR3M &  &  &  &  & -0.239815 &  &  &  &  \\
        L2.AED3M &  &  &  &  &  & -0.285123 &  &  &  \\
        L2.EIBO3M &  &  &  &  &  &  & -0.140055 & 0.055453 & 0.012622 \\
        L2.Oil & -0.002001 &  & -0.001964 &  & -0.000847 &  &  & -0.123944 &  \\
        L2.USDL3M &  &  &  &  &  &  & 0.181734 &  & 0.208902\\
        \hline
    \end{tabular}}%}      
\end{table}
\end{frame}

\begin{frame}{Results}
    \begin{table}
        \centering
        \caption{Granger Causality Test Results}
        \label{table4}
        \resizebox{\columnwidth}{!}{%
        \begin{tabular}{|cccc|}
            \hline
            Variable & Causality direction & Variable & p value\\
            \hline
            Oil & $\rightarrow / \leftarrow$ & AED3M & 0.11/0.67\\
            Oil & $\rightarrow / \leftarrow$ & BHD3M & 0.67/0.31\\
            Oil & $\rightarrow / \leftarrow$ & OMR3M & 0.00/0.03\\
            Oil & $\rightarrow / \leftarrow$ & QAR3M & 0.12/0.48\\
            Oil & $\rightarrow / \leftarrow$ & SAR3M & 0.00/0.00\\
            Oil & $\rightarrow / \leftarrow$ & KWD3M & 0.03/0.01\\
            Oil & $\rightarrow / \leftarrow$ & EIBO3M& 0.48/0.03\\
            \hline
        \end{tabular}}%}     
    \end{table}
\end{frame}

\begin{frame}{Evaluation}
\indent The results show that oil prices Granger-cause OMR3M, SAR3M, and KWD3M. The results also show that OMR3M, SAR3M, KWD3M, and EIBO3M Granger-cause oil prices. When we introduce covariates such as other countries' FX swaps into the model, the predictive power of oil prices on FX swaps is reduced or even negligible.

\indent The combined results further imply that the Granger causality relationship between oil prices and FX swaps is biased by the omitted variable, which is the FX swaps of other countries.
\end{frame}

\begin{frame}{Conclusion}
    \begin{itemize}
        \item The FX swaps have a significant impact on oil prices, but interest rates do not.
        \item The FX swaps of these countries, Oman and Qatar, impact oil prices the most.
        \item The powerful impact of Oman and Qatar's FX swaps on oil prices is a mystery that requires further investigation.
    \end{itemize}
\end{frame}


\begin{frame}{Reference}
\printbibliography
\end{frame}
\end{document}