\documentclass[12pt]{article}
\usepackage{diagbox}
\usepackage{amsmath}
\usepackage{graphicx} %插入图片的宏包
\usepackage{float} %设置图片浮动位置的宏包
\usepackage{subfigure} %插入多图时用子图显示的宏包
\usepackage{setspace}
\usepackage{xeCJK}
\usepackage{amssymb}
\usepackage[backend=biber, style=authoryear,]{biblatex}
\usepackage{enumerate}
\usepackage{multirow}
\usepackage[rightcaption]{sidecap}
\usepackage{caption}
\usepackage{fontspec}
\usepackage{booktabs}
\usepackage{amsthm}


%%%%%%%%%%%%%%%%%%%%%%%%%%%%%%%%%%%%%%%%%%%%%%%%%%%%%%%%%%%%%%%%%%%%%%%%%%%%
% \theoremstyle{definition}
% \newtheorem{definition}{Definition}[section]
% \newtheorem{theorem}{Theorem}[section]
% \newtheorem{corollary}{Corollary}[theorem]
% \newtheorem{lemma}[theorem]{Lemma}
% \newtheorem{prop}[theorem]{Proposition}


\newcommand{\matr}[1]{\mathbf{#1}} % undergraduate algebra version
%\newcommand{\matr}[1]{#1}          % pure math version
%\newcommand{\matr}[1]{\bm{#1}}     % ISO complying version



\setCJKmainfont{NotoSerifTC-Regular.otf} %自行去 google font 下載該字型
\XeTeXlinebreaklocale "zh"             %這兩行一定要加,中文才能自動換行
\XeTeXlinebreakskip = 0pt plus 1pt     %這兩行一定要加,中文才能自動換行
\defaultCJKfontfeatures{AutoFakeBold=6,AutoFakeSlant=.4} %以後不用再設定粗斜
\newCJKfontfamily\Kai{標楷體}           %定義指令\Kai則切換成標楷體
\newCJKfontfamily\Hei{微軟正黑體}       %定義指令\Hei則切換成正黑體
\newCJKfontfamily\NewMing{新細明體}     %定義指令\NewMing則切換成新細明體
\doublespacing
%%%%%%%%%%%%%%%%%%%%%%%%%%%%%%%%%%%%%%%%%%%%%%%%%%%%%%%%%%%%%%%%%%%%%%%%%%%%
\title{Further Exploration of "Oil, Foreign Exchange Swaps and Interest Rates in the GCC Countries"}
\author{Wei-Yu Fan\thanks{
Graduate student in Department of Economics, National Taiwan University.\\ 
Email address: entrencemania@gmail.com
}
} 
\date{May 2024}
\addbibresource{reference.bib}
%%%%%%%%%%%%%%%%%%%%%%%%%%%%%%%%%%%%%%%%%%%%%%%%%%%%%%%%%%%%%%%%%%%%%%%%%%%%
\begin{document}
\maketitle
\begin{sloppypar}
\begin{spacing}{0}
\begin{abstract}
\noindent 
In \textcite{almaskati2022oil} study, it was mentioned that there is a correlation between oil prices, foreign exchange (FX) swaps, and interest rates among the member countries of the Gulf Cooperation Council (GCC). The study found that oil prices indeed influence indicators such as forward exchange transactions, with Saudi Arabia showing the greatest spillover effect in forward exchange transactions among GCC member countries. Since the study covers the period from 2005 to 2020 and focuses on GCC member countries, there is a plan to extend it in two directions: incorporating data from 2020 to 2023 and including other energy like natural gas. The plan involves using a vector autoregression model (VAR) to regress oil prices, FX swaps, local interbank rates, SP500, gold prices, gas, and London interbank offered rates (LIBOR), and determining the lag period using Akaike Information Criterion (AIC) and the Schwarz Information Criterion (SIC). Subsequently, Granger causality tests will be conducted to examine the causal relationships within these variables.
\end{abstract}
\end{spacing}

\begin{tabular}{rl}
    \\
    \textbf{Keywords:} &GCC markets, oil, FX swaps, \\
    &interest rates, VAR model\\
\end{tabular}


\section{Introduction}
For oil-producing countries like the Gulf Cooperation Council (GCC), the majority of their foreign exchange comes from revenue generated by oil exports, making oil prices closely intertwined with foreign exchange swaps (FX swap). Additionally, since oil exports constitute a significant portion of GDP and government revenue, and government entities, public investment funds, and sovereign wealth funds are major liquidity providers for local banks, there should be a significant correlation between oil prices, FX swap, and interest rates. Therefore, the research question will focus on "whether oil prices affect FX swap and interest rates, and if so, how?"

This main reference, \textcite{almaskati2022oil}, investigates the relationship between oil prices, FX swaps, and local interbank offered rates across the six Gulf Cooperation Council (GCC) countries. It explores the potential hedging and diversification benefits of including oil in portfolios containing GCC FX swaps or interest rate instruments. The study finds that oil and oil-related exports are significant sources of USD liquidity and government funding in GCC economies, with oil prices predicting movements in GCC FX swaps and interbank offered rates. Specifically, the Saudi Arabian FX swap market has a notable influence on other GCC markets due to the country's dominant role in the region's economy.

This study contributes to the understanding of the relationship between oil and GCC FX swaps and interbank rates, offering insights valuable for trading and risk management decisions in the GCC market. Besides, we consider natural gas due to its complex relationship with oil. Natural gas, as a by-product of oil production, should exhibit a positive correlation with oil prices. However, at the same time, natural gas also serves as a substitute for oil in the energy market, as seen in recent decades with the emergence of technologies such as gas-powered vehicles and natural gas power generation.

\section{Related Literature}
Since this study was published in 2020, there are relatively fewer related studies. I have highlighted several studies that are highly relevant to it. \textcite{amano1998oil} observed that the trends in oil prices significantly impact the real exchange rate of the United States. \textcite{zhang2008spillover} noted a long-term correlation between oil prices and the US dollar exchange rate, suggesting that it is the US dollar exchange rate that influences oil prices. \textcite{reboredo2014oil} found a long-term contagion effect between oil prices and the US dollar exchange rate, exacerbating the 2008 global financial crisis. \textcite{eslamloueyan2015determinants} discovered that oil prices play a crucial role in determining the real exchange rates of GCC member countries in the long term. The original author, \textcite{almaskati2022wavelet}, employed wavelet analysis to examine the relationship between oil prices and forward exchange rates. This study found that the relationship between the two is small but significant at high frequencies, and strong and significant at low frequencies. Notably, there is no significant relationship between the two during periods of rising oil prices.

\section{Data}
We extracted the daily closing prices of several financial assets from Datastream during the period from May 2009 to May 2024. As GCC foreign exchange swap and/or interest rate data were not available before this period, we limited the time frame of our analysis. We calculated the continuously compounded daily returns as $\log \frac{P_t}{P_{t-1}}\times 100$, where Pt is the daily closing price. Unless otherwise stated, all analyses in our study are based on these returns. The variables extracted include: three-month forward foreign exchange levels for all GCC markets and the euro; the United Arab Emirates (EIBO3M), and the London Interbank Offered Rate (USDL3M); spot and futures commodity prices (oil/gold); the SP 500 U.S. stock index (SPX); and the MSCI Emerging Markets Index (MSCI-EM). As foreign exchange swap prices are affected by the actual term day count (three months in our case), which can vary from day to day due to holidays or calendar effects (e.g., 90 or 93 days), we adjusted the prices to be all based on 91 days. In addition, we used forward prices (i.e., foreign exchange spot prices adjusted by foreign exchange swap points) in our analysis, not just pure foreign exchange swap prices, as using the latter alone could distort the situation, especially in estimating daily volatility and return sizes. 

In this study, we refer to foreign exchange swap variables by their abbreviations, which represent the currency and term, such as AED3M (three-month UAE forward foreign exchange swap). Our analysis focuses on the three-month foreign exchange swap term, as this is the most liquid and actively quoted/traded term in FX swap. We also limited our analysis of local interbank lending rates to the rates in  the United Arab Emirates, as interest rate derivatives are only quoted for these two rates, and there is a lack of reliable and actively used interbank lending rates in other GCC markets. The addition of other international variables (USDL3M/gold/SPX/MSCI-EM) is for comparison of the role of oil versus these variables in hedging GCC foreign exchange swap and interest rate exposures.

\newpage
\section{Methodology}
We regress oil prices, FX swap, local interbank offered rates, SP500, gold prices, and LIBOR using a vector autoregression model (VAR). We determine the lag order using the Akaike Information Criterion (AIC) and the Schwarz Information Criterion (SIC), and then conduct Granger causality tests to examine the causal relationships between these variables.

Let , where $\matr{y}_t$ represents the vector of all variables, including changes in log FX swap, changes in log interbank rates, changes in log oil prices, and so on. The VAR model is represented as:
$$\matr{y}_t := (\Delta \log \text{FXswap}_t, \Delta \log \text{r}_t, \Delta \log \text{Oil}_t, ...)^T \in \mathbb{R}^p,$$
$$\matr{y}_t = \matr{A}_{-1}\matr{y}_{t-1}+\matr{A}_{-2}\matr{y}_{t-2}+...+\matr{A}_{-k}\matr{y}_{t-k}+e_t,$$
where $\matr{A}_{-k}$ is the coefficient matrix of lag k, $e_t$ represents the white noise error term.

\newpage
\section{Data Analysis}
In table \ref{table1}, we present the summary statistics of the variables used in our analysis. The table shows the mean, maximum, minimum, standard deviation, skewness, kurtosis, and the p-value of Jarque-Bera statistic, Augmented Dickey-Fuller (ADF) test, and Phillips-Perron (PP) test for each variable. The Jarque-Bera statistic tests the null hypothesis that the data is normally distributed, while the ADF and PP tests examine the null hypothesis that the data has a unit root. The results of these tests are all statistically significant, indicating that the data is not normally distributed and does not have unit root.
\begin{table}[h]
    \centering
    \caption{Summary Statistics of the Return rates}
    \label{table1}
    \resizebox{\columnwidth}{!}{%
    \begin{tabular}{lrrrrrrrrr}
        \hline
        Name & Mean & Maximun & Minimun & Std. Dev. & Skewness & Kurtosis & Jarque-Bera & ADF & PP \\
        \hline
        OMR3M & 0.000368 & 4.580394 & -3.126833 & 0.321510 & 1.127371 & 33.667119 & 0.000000 & 0.000000 & 0.000000 \\
        BHD3M & 0.000178 & 1.471405 & -1.590376 & 0.152445 & -0.123410 & 16.192026 & 0.000000 & 0.000000 & 0.000000 \\
        KWD3M & -0.000181 & 2.921430 & -2.821397 & 0.241882 & -0.075060 & 18.478365 & 0.000000 & 0.000000 & 0.000000 \\
        QAR3M & 0.000023 & 5.930881 & -5.955600 & 0.300399 & -1.029339 & 125.294164 & 0.000000 & 0.000000 & 0.000000 \\
        SAR3M & 0.000030 & 0.435031 & -0.461680 & 0.025713 & 0.564578 & 75.888848 & 0.000000 & 0.000000 & 0.000000 \\
        AED3M & -0.000012 & 0.295667 & -0.258926 & 0.012378 & 0.727567 & 171.852696 & 0.000000 & 0.000000 & 0.000000 \\
        EUR3M & -0.005693 & 2.567288 & -2.242401 & 0.527834 & 0.047230 & 1.589542 & 0.000000 & 0.000000 & 0.000000 \\
        EIBO3M & 0.019403 & 44.497508 & -49.061664 & 4.340056 & -0.578778 & 33.842150 & 0.000000 & 0.000000 & 0.000000 \\
        Gold & 0.023873 & 5.133427 & -9.596165 & 0.971210 & -0.455444 & 5.634502 & 0.000000 & 0.000000 & 0.000000 \\
        Oil & -0.020409 & 123.677917 & -305.966065 & 5.886999 & -33.463558 & 1913.033934 & 0.000000 & 0.000000 & 0.000000 \\
        MSCI EM & 0.011127 & 5.581817 & -6.943303 & 1.026245 & -0.399815 & 3.778718 & 0.000000 & 0.000000 & 0.000000 \\
        SPX & 0.045251 & 8.968316 & -12.765214 & 1.081019 & -0.707849 & 13.183011 & 0.000000 & 0.000000 & 0.000000 \\
        USDL3M & 0.047995 & 24.885484 & -27.263636 & 1.702867 & -0.801945 & 48.911574 & 0.000000 & 0.000000 & 0.000000 \\
        GAS & -0.032772 & 38.172675 & -69.314718 & 3.662205 & -1.330571 & 39.500504 & 0.000000 & 0.000000 & 0.000000 \\
        \hline
    \end{tabular}}%}
\end{table}

\begin{table}[ht]
    \centering
    \caption{Correlation of the Return rates}
    \label{table2}
    \resizebox{\columnwidth}{!}{%
    \begin{tabular}{lrrrrrrrrrrrrrr}
        \hline
         & OMR3M & BHD3M & KWD3M & QAR3M & SAR3M & AED3M & EUR3M & EIBO3M & Gold & Oil & MSCI EM & SPX & USDL3M & GAS \\
        \hline
        OMR3M & 1.000000 & &  &  &  &  &  &  &  &  &  &  &  &  \\
        BHD3M & 0.064175 & 1.000000 &  &  & &  &  &  &  &  &  &  & &  \\
        KWD3M & -0.004342 & 0.048295 & 1.000000 &  & & &  &  &  &  &  &  &  & \\
        QAR3M & -0.032026 & 0.014369 & 0.021553 & 1.000000 &  & &  &  &  &  &  &  &  &  \\
        SAR3M & 0.075297 & 0.038707 & 0.016946 & 0.024471 & 1.000000 & &  &  &  &  &  &  &  &  \\
        AED3M & 0.073529 & 0.018838 & 0.087568 & 0.002695 & 0.064088 & 1.000000 & &  &  &  &  &  &  &  \\
        EUR3M & 0.007711 & 0.010918 & -0.202587 & 0.010328 & 0.000726 & -0.025512 & 1.000000 &  &  &  & & &  &  \\
        EIBO3M & -0.011979 & -0.014450 & -0.010113 & -0.013793 & 0.013951 & -0.004707 & 0.021393 & 1.000000 & &  & &  &  &  \\
        Gold & 0.037293 & 0.004764 & -0.084953 & 0.002930 & 0.003426 & -0.016769 & 0.363884 & 0.023088 & 1.000000 & & &  & &  \\
        Oil & 0.058072 & 0.036930 & -0.018069 & -0.002890 & 0.082453 & 0.014696 & 0.049036 & -0.000047 & 0.045790 & 1.000000 &  &  & &  \\
        MSCI EM & -0.000011 & -0.000820 & -0.064095 & -0.016388 & -0.003269 & 0.012727 & 0.292853 & -0.004832 & 0.210766 & 0.100368 & 1.000000 &  &  &  \\
        SPX & 0.047305 & -0.009077 & -0.067865 & -0.029824 & 0.027595 & -0.007703 & 0.162321 & -0.009381 & 0.068460 & 0.138281 & 0.453554 & 1.000000 & &  \\
        USDL3M & 0.042462 & -0.031377 & -0.007134 & -0.033063 & -0.000350 & -0.014125 & -0.070792 & -0.018465 & -0.078611 & -0.023823 & -0.041639 & 0.009709 & 1.000000 & \\
        GAS & -0.012172 & 0.004382 & -0.025919 & 0.024487 & 0.017694 & -0.015924 & 0.024572 & -0.034489 & 0.000870 & 0.005898 & 0.042040 & 0.070901 & -0.025984 & 1.000000 \\
        \hline
    \end{tabular}}%}
\end{table}

In table \ref{table2}, we present the correlation matrix of the variables used in our analysis. We notice the presence of relatively higher positive correlations among the various commodity (oil/gold/gas) and equity markets (MSCI-EM and SP500).The table shows the correlation coefficients between each pair of variables. The results show that oil prices have a positive correlation with EUR3M, gold, and the SP500 index. The correlation between oil prices and FX swaps is relatively low, which is consistent with the findings of the original study. 

\section{Results}
We find out that the AIC and BIC results are quite different. The AIC suggest a lag of 9, while the BIC suggest a lag of 2. We choose the lag of 2 as it is more conservative. The results of the VAR model are presented in table \ref{table3}. The results show that the impact of lag 1 and lag 2 of OMR3M impact on oil price is significant, largest, and negative. Besides, SAR3M has largest and significant positive impact on OMR3M. Third, the impact of lag 1 of AED3M on KWD3M is significantly large. 
\begin{table}[ht]
    \centering
    \caption{VAR Model Results(5\%)}
    \label{table3}
    \resizebox{\columnwidth}{!}{%
    \begin{tabular}{llllllllllllllr}
        \hline
         & OMR3M & BHD3M & KWD3M & QAR3M & SAR3M & AED3M & EIBO3M  & Oil & USDL3M\\
        \hline
        L1.OMR3M & -0.388544 & 0.026061 & 0.035832 &  & 0.002620 &  &  & -0.773951 &  \\
        L1.BHD3M & 0.093795 & -0.415733 &  &  &  &  &  &  &  \\
        L1.KWD3M & 0.043305 &  & -0.258733 &  &  & -0.001989 &  &  & 0.320108 \\
        L1.QAR3M & 0.036584 &  &  & -0.445284 &  &  & 0.469332 &  &  \\
        L1.SAR3M & 0.584286 & 0.256249 &  &  & -0.428796 & 0.016081 &  &  &  \\
        L1.AED3M &  &  & -1.053435 &  &  & -0.559541 &  &  &  \\
        L1.EIBO3M & -0.002523 &  &  &  &  &  & -0.367572 &  & 0.027283 \\
        L1.Oil &  &  &  &  &  &  &  & -0.253198 &  \\
        L1.USDL3M &  &  &  &  &  &  & 0.204014 &  & 0.132099\\
        L2.OMR3M & -0.092432 &  & 0.039735 &  &  &  & -0.463412 & -1.363542 &  \\
        L2.BHD3M &  & -0.171601 &  &  &  &  &  &  &  \\
        L2.KWD3M & 0.048206 & 0.021473 & -0.093834 &  & 0.004291 &  &  &  &  \\
        L2.QAR3M &  &  &  & -0.159426 &  &  &  & -0.854648 &  \\
        L2.SAR3M &  &  &  &  & -0.239815 &  &  &  &  \\
        L2.AED3M &  &  &  &  &  & -0.285123 &  &  &  \\
        L2.EIBO3M &  &  &  &  &  &  & -0.140055 & 0.055453 & 0.012622 \\
        L2.Oil & -0.002001 &  & -0.001964 &  & -0.000847 &  &  & -0.123944 &  \\
        L2.USDL3M &  &  &  &  &  &  & 0.181734 &  & 0.208902\\
        \hline
    \end{tabular}}%}      
\end{table}

In table \ref{table4}, we present the results of the Granger causality tests. The results show that oil prices Granger cause OMR3M, SAR3M, and KWD3M. The results also show that OMR3M, SAR3M, KWD3M, and EIBO3M Granger cause Oil price The results suggest that oil prices have a significant impact on the FX swaps and interest rates of half of GCC countries, and that the FX swaps and interest rates of these countries also have a significant impact on oil prices.

\begin{table}[ht]
    \centering
    \caption{Granger Causality Test Results}
    \label{table4}
    \resizebox{\columnwidth}{!}{%
    \begin{tabular}{lllr}
        \hline
        Variable & Causality direction & Variable & p value\\
        \hline
        Oil & $\rightarrow / \leftarrow$ & AED3M & 0.11/0.67\\
        Oil & $\rightarrow / \leftarrow$ & BHD3M & 0.67/0.31\\
        Oil & $\rightarrow / \leftarrow$ & OMR3M & 0.00/0.03\\
        Oil & $\rightarrow / \leftarrow$ & QAR3M & 0.12/0.48\\
        Oil & $\rightarrow / \leftarrow$ & SAR3M & 0.00/0.00\\
        Oil & $\rightarrow / \leftarrow$ & KWD3M & 0.03/0.01\\
        Oil & $\rightarrow / \leftarrow$ & EIBO3M& 0.48/0.03\\
        \hline
    \end{tabular}}%}     
\end{table}

\section{Conclusion}
In this study, we examive the relationship between oil prices, FX swaps, and interest rates in the GCC countries. We find that FX swaps and interest rates of part of the GCC countries have a significant impact on the oil prices, and the interest rates of these countries also have a significant impact on oil prices. These country are Oman and Qatar which are the 15th and 20th largest oil producers in the world. The results is quite different from the original study which found that oil price has impact on more country like Baharian and Kuwait. The results suggest that the impact of oil prices on FX swaps and interest rates in the GCC countries is more complex than previously thought. The powerful impact of Oman and Qatar on oil prices is a mystery that requires further investigation. One suspectable reason is that the oil price is more sensitive to the political status of these country. For example, the civil war in Yemen which is near Oman may have a significant impact on Oman and global oil production. Another reason is that our dataset is quite different from the original study. The scale of daily return of FX swap in original study is often 10 times larger than our dataset. This may lead to the difference in the results.


\section{comment}
As mentioned in the Abstract, this proposal will adopt an approach that extends the time and spatial scale, incorporating data from 2020 to 2023, and also includes data like gas. The exchange rates in the data will be based on three months (91 days), and all data will be processed as $100 \times \log \frac{P_t}{P_{t-1}}$. All of the above data will be obtained from Datastream.

\printbibliography
\end{sloppypar}
\end{document}